\section{Literature Survey}
\label{sec:litsurvey}
We now present a brief, non-exhaustive, survey of virtual laboratory
work reported in the last couple of years.


The work reported in \cite{carlos} allows multiple users the access to
experimental and computational resources through only a Web browser, a
low-bandwidth Internet connection, and low user-side technical
specifications.  The server side is scalable and reproducible since it
is based on Linux operating system.  This solution makes use of a
Matlab license.   

The work reported in
\cite{aldo} can be used as a portable telelaboratory, completely
contained inside a Live Linux-RTAI DVD, which can be connected to a
large number of real plant to perform a process supervision task. 
This lab architecture makes use of RTAI-Linux and COMEDI and
alogorithms could be implemented in Matlab or Scilab.  

The work reported in \cite{odeh} allows the student to connect
components and equipment such as capacitors, resistors, transistors,
function generators with a switch system of a lab server and to map it
to a configuration data structure.  At run time, the GUI gets
constructed automatically, using Adobe Flash CS3.

The work reported in \cite{gabriel} is based on a PLC and ATmega8
microcontroller for data acquisition and control.  
The PLC is programmed using OMRON CX programmer 5.0 and the
microcontroller is programmed through AVR Studio 4.0.  The software
component is based on Java and MySQL.  Live video is a part of this
solution. 

The work reported in \cite{alberto} provide remote access to a
thermal control plant and a velocity control plant.  The former is
built with off-the-shelf components (some transistors and temperature
sensors), whereas the latter uses standard LEGO elements and minimal
electronics.  This work makes use of LabVIEW.

