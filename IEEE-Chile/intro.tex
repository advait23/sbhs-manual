\section{Introduction}
\label{sec:intro}
The objective of this work is to provide control laboratory
education to students who may not have access to good infrastructure.
This can be further detailed into the following specific objectives:
\begin{enumerate}
\item Provide a low cost laboratory.  One should be able to create,
  operate and maintain the laboratory at a low cost.  This requirement
  can be further divided into two sub-requirements:
\begin{enumerate}
\item Low cost hardware
\item Free software
\end{enumerate}
\item Provide remote access so that only Internet access is required
  at the student end.
\item Provide good learning experience:  It is not enough if the
  students enter parameters of a control algorithm, for example, P, I,
  D parameters;  they should be able to develop and validate their own
  control algorithm.
\item Scale the solution so that a large number of students can access
  the laboratory facility simultaneously.  This is important in the
  Indian context that has about a million students enrolling in
  engineering studies every year.
\item Access to a large number of students should be made with minimal
  infrastructure.  For example, the computer server and the Internet
  Protocol (IP) address should be able to accommodate many apparatus
  simultaneously.  This will allow minimally endowed Centres also to
  replicate such a facility and thus help evolve a distributed
  solution.  
\end{enumerate}
The issues raised in points 1(a) and 2 have been addressed in
\cite{ia010,vlabs-kmm} through a single board heater system (SBHS).
In the current work, we address all the remaining issues.  We begin
with a brief literature survey.  


% Hands-on experimentation is an important learning component for
% engineering education.  However, there is a lack of proper
% infrastructure, especially in remote areas, in India. Additionaly, the
% existing expensive lab equipments impose restrictions on type and
% duration of the experiments that could be done by the students.  This
% paper depicts a low cost single-board heater system virtual lab to
% address these issues. 

% Section 2 briefly discusses the features of the low cost, open source,
% single-board heater system (SBHS).  The low cost of a device is not
% enough to attract academia.  We explore the possibility of using this
% $\$50$ device as a part of virtual lab. 

% Though the web based labs have been in practice in a few countries for
% quite some time their usage in India is uncommon.  Section 3 explains
% the evolution of virtual labs for SBHS. Section 4 depicts the current
% hardware and software architecture of virtual SBHS lab at IIT Bombay.
% This lab allows the access of a single-board heater system, a control
% setup, through internet.  The control experiments performed on this
% setup, over internet, are described in Section 5. 
